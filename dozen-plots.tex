\documentclass[12pt]{article}

\newcommand{\litem}[1]{\item{\large #1} \nopagebreak}
\newcommand{\eg}{Examples: }
\newcommand{\book}[1]{\textit{#1}}
\newcommand{\story}[1]{``#1''}

\begin{document}

\title{The Dozen Plots} 
\date{} 
\maketitle

\begin{enumerate}
\setlength{\itemsep}{3mm}
\setlength{\parsep}{2mm}

\litem{Boy Meets Girl}

Any plot involving a romantic interest of one kind or
another, most typically male-female. By far and away the
most common plot complication.

\eg \book{Romeo and Juliet} by Shakespeare, \story{Hills
Like White Elephants}



\litem{Family Matters}

Plots involving a conflict between or among family members.

\eg \story{Sonny's Blues} by Baldwin, \story{Why I Live at
the P.O.} by Welty.



\litem{Buddy Stories or Rival Stories}

Stories involving relations or conflicts
between non-relatives without an explicit romantic angle.
The ``love-triangle'' story is a variation and combination of
this plot and ``boy meets girl.''

\eg \story{The Cask of Amontillado} by Poe



\litem{Internal Conflict}

A story where the protagonist's main problem is mostly
self-generated, as in a person trying to overcome mental
illness, his or her past, a drinking problem, or something
as basic as self-doubt.

\eg \story{Babylon Revisited} by Fitzgerald



\litem{Journey or Quest}

Some sort of travel -- literal, spiritual, or both -- is
involved. ``Coming of age'' stories are within this category.
Some critics argue that ``the quest'' informs all fiction.

\eg \book{On the Road} by Kerouac, \book{Huck Finn} by Twain



\litem{Encountering the Other; Dealing with Difference}

Plots where one person has to deal with someone he finds
alien or different somehow, such as a person from a
different social class, race, religion, etc.

\eg \book{The Prince and the Pauper} by Twain, \book{The
Chosen} by Potok



\litem{Encountering the Unusual}

Unlike the previous, in this category, the ``other'' is not
just a normal human being that the protagonist just hasn't
had a change to meet yet, but a genuinely other-worldly sort
of thing, such as a ghost or a monster.

Other plotlines will work into this, but the key question is
how to deal with this strange thing that has suddenly
entered our lives.

\eg \story{A Very Old Man With Enormous Wings} by Garc\'ia
M\'arquez, \story{Enormous Radio} by John Cheever.



\litem{Issue Stories}

Though many stories have issues, ``issue stories'' are those
in which the issue predominates -- stories making a point
about abortion, religious liberty, political corruption,
etc.

\eg \story{The Ones Who Walk Away From Omelas} by Ursula
LeGuin



\litem{Man vs. Nature}

Stories where humans are pitted against natural forces, be
they hurricanes or sharks or cold.

\eg \story{To Build a Fire} by London



\litem{Individual vs. Society}

One against many: One person brave enough to stand up
against a town's injustice, or an ``outsider'' story.

\eg \story{The Mysterious Stranger} by Twain, \story{The
Legend of Sleepy Hollow} by W. Irving



\litem{Faux Essay Story}

There are number of fictions in which a writer purports to
write an essay, but it is a fiction because the subject of
the essay is wholly or partly fictitious.

\eg \story{Remembery Fidelman} by Woodie Allen, a comic
parody of a eulogy for a great man; \story{The Approach to
Almotas\'im} by Borges, a book review of a nonexistent
book; \story{The Thousand and Second Tale of Scheherazade}
by Poe, which also fits into ``variations''



\litem{Variations on Existing Plots}

Only a separate category because it is so often done and is
a certain way to find a topic for a story, this number
encompasses all those stories which are conscious attempts
to retell, change, or parody stories already published or
well known.

\eg \story{The Real Story of the Three Little Pigs} by A.
Wolf, a well known ``children's'' book where the wolf gets to
tell his side of the story; \story{West Side Story}, a
conscious reworking of \book{Romeo and Juliet}, which of
course was also borrowed from earlier sources



\end{enumerate}




\end{document}

